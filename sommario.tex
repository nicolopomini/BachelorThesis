\chapter*{Abstract} % senza numerazione
\label{summary}
\addcontentsline{toc}{chapter}{Abstract} % da aggiungere comunque all'indice
Nowadays digital marketing heavily relies on the segmentation of an audience at both aggregated and at the single user's level. A life event is a very important event occurring in someone's life, such as a wedding or the birth of a child. It may have a relevant impact on the person's life, guiding many of her choices in terms of purchases, services and mobility. If identified, life events can represent a huge business opportunity and this is why they are extremely relevant for marketeers. 

The best and easiest places to look for life events are social media: they have become one of the most used means of communication, where people share daily news, their opinions and personal life updates. The catchment area of these services has grown exponentially by size in the last few years: for example Facebook has more than 2.2 billion monthly active users as of January 2018\footnote{as reported by Wikipedia, \url{en.wikipedia.org/wiki/Facebook}}.

Life event identification on social media is an open problem; there is no software solution that detects this kind of information into a social timeline. Consequently, the research question concerns the feasibility of the problem.

This thesis was developed at U-Hopper\footnote{\url{u-hopper.com}}, a small enterprise specialized in big data analytics solutions, and the detection of life events represents for it a concrete need and a realistic future business.

The proposed solution goes a step forward with respect to the state of the art, offering a detection method based on the entire user's timeline on social media. The system that is presented uses a hybrid technique to perform its scopes: a machine learning-based method to make post classification, and a model based approach to analyze the timeline. The system is designed to work with different social networks. It does not use any text or picture analysis technique, but a method based on the Wikipedia entities, to understand the content and the meaning of photos and texts. A prototype was developed and evaluated, to understand how what has been designed behaves in reality. The machine learning part reaches the benchmarks of the state of the art, making a post classification with both precision and recall measures around 0.9, while the timeline analysis offers an accurate estimation of the period when the life event occured (in 64\% of cases the life event is identified in a week range), and does a good discrimination on those people who have not lived any life events.

Finally, a new open dataset about life event classification on Twitter has been released, containing more than 5000 samples written by 45 users, with the purpose of simulating a real social timeline.
