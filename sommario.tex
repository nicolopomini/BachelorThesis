\chapter*{Summary} % senza numerazione
\label{summary}
\addcontentsline{toc}{chapter}{Summary} % da aggiungere comunque all'indice
Everyone's life is full of events, some of these are daily routines, some are minor and not too much attention is given to them, and some others are real milestones. The latter ones are called \emph{life events}: according to Cambridge Dictionary\footnote{\url{dictionary.cambridge.org}}, they are \textit{"a very important event in someone's life, such as marriage, the birth of a child, or the death of a family member"}. These kind of events are quite rare in a lifetime, and they may bring with them a big charge of stress, and big changes for those who live it. Furthermore, they don't last only for the day they happen, but there is a medium or long period that is influenced by the event: for example, a pregnancy lasts for 40 weeks, or a wedding is usually organized in several months; but also a negative life event, like the death of someone dear, causes bad feelings for a period more or less long for those who live it. The Holmes and Rahe stress scale \cite{holmes1967social} puts in relation several life events for the load of stress they cause: on a scale from 0 to 100, the most stressful event for an adult is the death of a spouse, that scores 100, but also positive events are part of this list, such as marriage, with the score of 50, and a pregnancy, with 40. In addition, a life event can be followed by another one life event: for example in Italy, 80\% of births occur within a marriage in 2012\footnote{\url{www.istat.it/it/files//2015/02/Avere_Figli.pdf}}.

Therefore, a life event is a signal of many factors in who lives it. If it is detected in time, it can be a great opportunity for many entities, such as banks, insurance companies and other kinds of ad-hoc marketing campaigns. The life events have a deep effect on the individual's spending habits and purchase patterns. According to the results\footnote{\url{travelbehaviour.files.wordpress.com/2014/06/lttb_carownbriefingnote_16-june.pdf}} of a study made by the University of the West of England about the relationship between life events and travel behaviour \cite{chatterjee2015facts}, households are more likely to change the number of cars at the time of life events: the "\textit{Birth of a child increases likelihood of a non-car owning household acquiring a car and increases likelihood of a two-car owning household relinquishing a car. This suggests households seek a one car solution when having children}".

Nowadays one of the most common way to make announcements or to share something about private life is via \emph{social networks}. A social network is a web application that allows users to create their own network of friends and relationships, sharing their news feed and reading that of others. The 
catchment area of these services has grown exponentially by size in the last few years: for example Facebook has more than 2.2 billion monthly active users as of January 2018\footnote{as reported by Wikipedia, \url{it.wikipedia.org/wiki/Facebook}}. Social networks are compared with the industrial media, like televisions and newspapers, because they are widely used to share information. For this reason they are also called \emph{social media}.

This thesis work makes a move forward to the actual state of the art about life event detection on social media: nowadays the literature does not offer any kind of solution that analyzes a user's timeline for this purpose. The system that is presented is able to detect a life event into someone's social timeline, using a hybrid technique to perform its scopes: a machine learning-based method to make post classification, and a model based approach to analyze the timeline. The system is designed to work with many social networks, and to inspect both texts and photos. It does not use any text or picture analysis technique, but a method based on the Wikipedia entities, to understand the content and the meaning of photos and texts. A prototype was developed and evaluated, to understand how what has been designed behaves in reality. Providing the ID of the user to analyze on a given social network, and a life event to search for, it takes care of every step needed to look for the event into user's social timeline: it deals with content downloads, content classification and user's timeline analysis, returning a list of time ranges, one for each identified event.

Finally, a new freely available dataset about life event classification on Twitter has been released, containing more than 5000 samples written by 45 users, and which is very unbalanced, like a real social timeline.

