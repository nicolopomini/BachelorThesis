\chapter{Introduction}
\label{cha:intro}

This thesis work was developed during a 3 month internship at U-Hopper\footnote{\url{u-hopper.com}}, with which I could combine a work experience in a company with the writing of the bachelor's thesis. U-Hopper is a small business which develops IoT and big data analytics solutions, offering products for user profiling and for citizen engagement and working with public administrations. They are active in several research fields - for example in fog computing - and they deal with many european projects in the field of technology. Life event discovery can be a significant business for the company, because it is already into the user profiling world, but above all because detect a life event into someone's life can bring many opportunity with banks, insurances, estate agents and many other businesses, which are constantly looking for such a meaningful information like a life event can be.

A life event is a kind of event that is rare into someone's life, and which brings with it many actions and many consequences. It is a life changing milestone for everyone, which implies new habits and new needs for who lives it: thinking about the birth of a child, a family may need a bigger car or a bigger house, a life insurance for the baby, or even a new job position with different working time. For these reasons, many entities can use a life event discovery to create ad-hoc advertising campaign. 

The identification of a life event into social media contents is a \emph{machine learning} problem, and to be more precise, a classification one. Machine learning is a field of computer science which deals with allowing a computer system to "\textit{learn with data, without being explicitly programmed}" \cite{samuel1959some}. It can be applied in many contexts, such as taking decisions, or make optimizations, forecasts and predictions. Nowadays a human being faces itself with machine learning in everyday life: home assistants, security surveillance, music and shopping suggestions, customer services are strongly powered by artificial intelligence. These services relies on data to learn how to work as good as possible: they are trained with samples of data similar to what they expect to receive by their users: the more accurate, exhaustive and in large quantities they are, the better the system learns. Therefore, data have a very central role in machine learning problems.

A classification task has the goal of assigning a belonging class to a given object. The input is composed by a tuple of \emph{features} that characterize the object, usually made by numbers, and the output is a categorical variable, such as a "yes/no" label. In other words, it can be seen as a mathematical function, that maps a vector $ \boldsymbol{x} \in \mathbb{R}^n $ to an answer $ y \in C $
\begin{gather*}
f \colon \mathbb{R}^n \to C \\
f \colon \boldsymbol{x} \mapsto y
\end{gather*}
where $C$ is a set of possible categories. A famous educational example of this kind of problem is the Iris flower classification\footnote{\url{en.wikipedia.org/wiki/Iris_flower_data_set}}, where the input $ \boldsymbol{x} $ is composed by the sepal and petal widths and lengths in cm of the flower, and $C = \{\text{\texttt{Iris setosa}, \texttt{Iris virginica}, \texttt{Iris versicolor}}\}$. For each sample composed by 4 measures the belonging class is predicted.

The background of live event identification on social media consists in a huge stream of documents, called \emph{posts}, each one written by a specific user and containing text, images, external links and other attachments, with a date of publication. Every user can share, comment or like someone else's post. A \emph{timeline} is the list of posts written by a single user, sorted by decreasing date of publication. These data are freely available on the Web (in some cases the author's authorization is necessary to access it) and contain many information. There is an endless number of works that use social media data for various purposes, and in this case they are used to detect events.

\section{Research objectives}
Life event detection on social media is still an open problem, in fact such a commercial solution to be integrated into a social system of some commercial, banking or insurance entity does not exist, neither an accademic pubblication that goes beyond a classification problem. Therefore, the reseach question is about the feasibility of this kind of problem.

The goal of this thesis is to find a way to understand whether a person have lived a life event, or if she is about to live it, observing her activities on social media. Nowadays, social media are widely used all over the world: according to the Global Digital 2018 report\footnote{\url{wearesocial.com/it/blog/2018/01/global-digital-report-2018}}, 3.196 billions of people use social media every month, sharing messages, photos and other contents that are also about their private life. They are used to deliver messages of congratulations or support, to search for information from users who have experienced certain situations or to make announcements of news or changes. Consequently, social media are one of the best platform to get reliable information about people in an easy, fast and free way.

The idea is to design a hybrid system that combines a data driven approach to analyze each post, and a model driven solution to inspect a user's timeline, in order to find out whether the user has lived a life event according to what she shared on social media. This work is focused on the detection of two life events, marriages and births of a child.

\section{Detection vs prediction}

A user's timeline is composed by all the posts published from the time she signed up to the social media to today. From this data is possible to \emph{detect} an event that happened in the past. Event detection is the identification of an event that occurred, into a stream of data, and now it is finished or at most it is still ongoing. Another possibility is to \emph{predict} an event, understanding whether the event can occur in a more or less near future. Event prediction can be done in several ways, for example observing what happen in the user's past, or in timelines of users with similar age, location and preferences, or using some logical model: for example, a married person will likely have a baby in her lifetime.

Of course, the two things have a very different meaning. With event detection is possible to understand what a person lived in the past, what she might searched or needed at the time of the event, and which are her usual needs of a person: for example, if it's known that a user had a child, her possible future vehicle will be a family car, or her new house should probably have at least two bedrooms. With event prediction is possible to anticipate some user's choices or needs, giving her some ad-hoc offers and advertising campaign: for example, if it was known that a user would probably have a child in the near future, she would likely look for a new bedroom furniture or a new bigger car.

From the machine learning point of view there is no big differences betweet event detection and event prediction. In both cases the usual patterns that indicate that the event is about to occur will be the same. The only difference is that the detection is based on facts, and so it's possible to say whether the obtained result is correct or not, while prediction on suppositions, which is not always verifiable. This thesis will be focused on life event detection, so on what the user has lived.

\section{Outline}

This thesis work is organized as follows: in chapter 2 the state of the art will be introduced, explaining what is missing for this purpose and what has to be changed. In addition, the data situation about life event will be shown, unfolding a solution that is different from the standard text classification. In chapter 3 the design solution will be presented, with a particular focus on the logical behind the system and on the used algorithms. Chapter 4 will be about the implementation procedure and the performance evaluation of a prototype of the system, and finally, in chapter 5 will be presented some final remarks and observations for future developments.

