\chapter{State of the Art}
\label{cha:intro}

Event detection using social networks is already a common practice. The literature offers many example of global scale event detection based on the analysis of social media contents, such as real-time earthquake identification based on tweets \cite{sakaki2010earthquake}, breaking news discovery in Twitter \cite{jackoway2011identification, phuvipadawat2010breaking}, or big gigs recognition observing what is posted on Flickr \cite{liu2011using}.

Event prediction using social network contents is also pretty common: the most striking example of the last few years are the 2016 american elections. In fact, while many of the official polls made by the most famous american newspapers and televisions had always forecasted Hillary Clinton as winner, social media reactions had been increasingly in favor of the Republicans during the election campaign \cite{elections}, and the rest is history. Other cases of event prediction using social media are, for example, movie box-office \cite{asur2010predicting} or Oscar-winner forecasts.

There is much less literature about life-event detection, which will be briefly described in the following section.

\section{Life event detectors}
\label{sec:classifiers}
The core of the literature is focused on the creation and the performance evaluation of classifiers, which role is making a life event identification in textual contents taken from social networks. All the models proposed in scientific papers and articles follow a similar pattern. Firstly, a set of life events to search for is fixed (e.g. getting married, birth of a child, buy a new house, etc.), then data is fetch from social media, analysed or labelled, and finally a machine learning classifier is trained and tested on that data. Each of previous steps is executed in several ways, in particular:
\begin{itemize}
\item Data is downloaded following two different philosophies. Some authors prefer to fetch only contents that contain specific keywords (such as "\textit{engagement}" for marriage) \cite{dickinson2015identifying, cavalinclassification, moyanolife, khobarekar2013detecting}, labelling each content \textit{by hand} as about or not to the life event itself; some other search for contents randomly, considering a text related to the life event if it contains at least one keyword \cite{choudhury2014personal, di2013detecting}.
\item Feature estraction is done mainly in two ways - working with the text itself, using bag-of-words or bag-of-N-grams \cite{cavalinclassification, di2013detecting, li2014major} - using a semantic analyzer, to obtain sentiment, formality with which the text is written, Wikipedia entities and topics contained, etc \cite{khobarekar2013detecting}. In addition to that, some models consider also "external" features, like user's features (number of friends, age, location, post frequency, etc.) or post success \cite{dickinson2015identifying}.
\item The majority of models consider only a single post at a time, while few others \cite{cavalin2015multiple, moyanolife} consider also the reactions and the conversation related to a main post.
\end{itemize}

For our purpose, these cited models have several weaknesses. First of all, they are only classifiers that tell if a post is about or not to a life event, they don't go further on author history. Furthermore, there is no time consideration: a post can be about something in the past, in the present or in the future, and is completely isolated from other posts. They are limited to the detection only, there is no perception of how much the event may have lasted. In addiction, a post related to a life event may not be concerned with the author (e.g. participate to a wedding instead of getting married). Last but not least, all these models are focused only on Twitter, and only on texts.

\section{A step further}
\label{sec:further}
The model described by Cavalin, Gatti, Pinharez for the \textit{IBM InfoSphere BigInsights} platform \cite{cavalin2014towards} goes a step further to all the previous classifiers. In fact, starting from life event detection on a sample of tweets, it performs a user entity matching on an existing database of clients, aimed to decide which is the best approach to offer them some services or some products. It solves the reverse problem of ours: its goal is to get a list of users that posted about a life event in a given time window. 

To do that it applies fistly a slack word matching filter to download data from Twitter, then it uses more complex rules, like combination of words to make a further filter, and finally, using a machine learning classifier, it decides if a tweet is related to the life event, and consequently if the user is interesting. Once the user is detected, his/her information are used to match him/her into a database of clients.

In common with our model there is a move forward from a simple classifier, but with a big difference: they take a wide range of tweets from all over the world to analyse the presence of a life event, not considering user's timeline at all, while our model is focused on a single user analysis. Furthermore, this model is also concentrated only on Twitter contents.

\section{Other interesting methods}
\label{sec:other}
In addition to the previous models, the literature offers several works that are methodologically interesting. They will be briefly described now.

As already mentioned, life event detection is a branch of event identification. The core of the business is therefore the identification of events from a stream of time-stamped documents coming from social networks. In concrete, the job made by Vavliakis, Symeonidis and Mitkas \cite{vavliakis2013event} organizes documents from various sources according to the event they describe, assigning to the event an importance, while the one made by CC Chen, MC Chen and MS Chen \cite{chen2009adaptive} tracks how much activity is related to an event from a global stream of documents. However, these two models work on global scale, not analyzing a specific user, and they have an implicit issue: they consider the importance of the event based on the \textit{noise} the event brings with him. The more people talk about an event, the more important it is. Therefore every global event, such as The Olympics, will be classified as much more important than a wedding or a birth of a child.

Another interesting study is the one proposed by Li, Ritter and Jurafsky \cite{li2014inferring}, a model-driven system to infer user's attitudes or preferences reasoning over user's attributes and social network graphs. It builds, for every person taken into analysis, a series of predicates for his/her attributes (location, gender, education), relationships (married, friends) and preferences (what he/she likes/dislikes), which can be used to detect important events on his/her timeline or in friends/relatives social profiles. This work is interesting beacuse is the only one that is not completely driven by machine learning, but it offers a logical model, and also it combines multiple social networks (Twitter and Google+). On the other hand, of course, this model is not designed to deal with life events.

A last interesting study is about topic sentiment analysis using the hashtah graph in Twitter \cite{wang2011topic}. It demonstrates that hashtags offer additional information to texts, classifying tweets and users in categories: the use of a specific hashtag may be connected with a specific life event. This kind of study, however, has more sense on global scale analysis, like searching for users who show interest in a specific topic: it may be not so useful in a context of life event detection, which is concentrated on a single-user analysis.